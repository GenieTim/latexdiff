\documentclass{ctexart}
% needs compilation with xelatex
\begin{document}
Chinese: 汉汉汉汉字汉汉汉汉

Hiragana: ああああいああああ

Katakana: アアアアイアアアア

例えばS波の到達時刻を知りたい場合,ジェフェリーズとブレンの掃除表が使えますか?


1912年にHessによって宇宙線が初めて発見されて以来、広いエネルギー範囲、多種多様な検出器によって宇宙線観測が行われてきた。また、ガリレオ以来発達してきた可視光での天体観測も、電波望遠鏡や赤外望遠鏡という新しい観測手段の登場により、多波長観測へと発展した。

宇宙線といっても、その成分は陽子、原子核、電子、ニュートリノなど様々であり、そのエネルギー範囲も何桁にもわたる。現在、地上で確認されている宇宙線のうち、最もエネルギーの高いものは$10^{20}$~eVを超える(いわゆる最高エネルギー宇宙線)。これは加速器で人類が到達できるエネルギーを8桁も上回るが、なぜそのような高いエネルギーの宇宙線が存在するのかは解明されていない。宇宙線の加速機構、地球までの伝播過程、また1次宇宙線成分は何であるのかは、いずれも未解決の問題であり、将来の宇宙線観測計画による解決が期待される。

\end{document}
